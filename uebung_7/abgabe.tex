\documentclass[12pt, a4paper, titlepage]{article}

\usepackage{ngerman}
\usepackage{lmodern}
\usepackage{amssymb }
\usepackage{subfig}
\usepackage{colortbl}
\usepackage{graphicx}
\usepackage{url} 

\title{Messtechnik und Messdatenverarbeitung \\ \"Ubungszettel 7}
\author{Lasse Knudsen (21157556), Maximilian Scholz (21158423), \\
	Florian Wiesener (21155905)  \\
	Technische Universit\"at Hamburg-Harburg \\
}


\date{\today}
\begin{document}
	\maketitle
	 \section{Aufgabe 1}
	 \label{sec:a1}
	 
	\subsection{Welchen der behandelten Hypothesentests k\"onnen sie einsetzen, um eine Abl\"osung des Sensors zu
	 	detektieren? Beschreiben Sie kurz wie und unter welchen Voraussetzungen der Test anwendbar ist?}
	Wir k\"onnen den Parametertest benutzen. Damit der Parametertest angewandt werden kann m\"ussen die Messwerte unabh\"angig sein und die Grundgesamtheit muss normalverteilt sein. Au\ss erdem muss ein Erwartungswert f\"ur die Grundgesamtheit existieren.
	
	\subsection{Bitte ordnen Sie die Messwerte den jeweiligen Klassen zu. Ist diese Einteilung sinnvoll?}
	
	Siehe 7.1\_hypothesentest.R \\
	Eine bessere Aufteilung der Klassen w\"are, eine Klasse zu erstellen, die den Mittelwert nicht nur beinhaltet sondern ihn gleichm\"a\ss ig umschlie\ss t. Also sowohl in positive als auch negative Richtung gleich weit reicht. Von dieser ersten Klasse ausgehen die weiteren Klassen symmetrisch anlegen. Dadurch, dass der Mittelwert nicht mittig in einer Klasse liegt, sollte sich das Histogramm verschieben.
	Au\ss erdem gibt es L\"ucken in den Klassen. Die erste m\"usste zum Beispiel $\leq$35.1 sein.
	
	\subsection{Bestimmen Sie die zu erwartenden H\"aufigkeiten.}
	Von links nach rechts:\\
  4.931936 \\
  9.635907 \\
  15.972651 \\
  19.597807 \\
  17.799406 \\
  11.966329 \\
  5.148013 \\
  3.736640 \\
  F\"ur die Rechnung siehe 7.1\_hypothesentest.R
	
	\subsection{Bestimmen sie die Pr\"ufgr\"o\ss e und ermitteln Sie das zugeh\"orige Vertrauensniveau.}
	Da die Varianz der Grundgesamtheit nicht gegeben ist, wird diese durch die Stichprobenvarianz gesch\"atzt.
	Die Pr\"ufgr\"o\ss e ergibt sich folgenderma\ss en:
	$$t=\frac{|x - \mu_0|}{s_x}\sqrt{n} = c$$
	F\"ur $\mu_0$ benutzen wir den Erwartungswert der Messwerte.\
	Es ergibt sich: 
	$$c = 1.102112$$ und $$p(c)\approx 0.73$$ (Messtechnik
	Systemtheorie f\"ur Ingenieure und Informatiker, Fernando Puente Leon, Uwe Kiencke, Tabelle A.1)\\
	F\"ur die Rechnung siehe 7.1\_hypothesentest.R 
	
	\subsection{Interpretieren Sie das Ergebnis.}
	Da $\alpha$ normalerweise im Bereich von $0.001 \leq \alpha \leq 0.05$ gew\"ahlt wird, k\"onnen wir die Nullhypothese annehmen, da $P(c)\leq 1-\alpha$ f\"ur alle so zugelassenen $\alpha$ gilt.
	 
	 \section{Aufgabe 2}
	 
	 \subsection{Bitte erl\"autern Sie die Bedeutung der Toleranzgrenze.}
	 Die Toleranzgrenze gibt an, innerhalb welchen Intervalls um das Sollma\ss  das Ma\ss  eines Werkst\"ucks liegen muss,
	 um die gesetzten Qualit\"atsstandards zu erf\"ullen.
	 
	 \subsection{Berechnen Sie die Standardabweichung und die Abweichung von der Toleranzfeldmitte.}
	 Siehe 7.2.2\_abweichung.R.
	 
	 \subsection{Zeichnen Sie schematisch das Toleranzfeldund die Verteilung deer Bauteill\"angen. Markieren Sie die
	 	Toleranzfeldmitte und die Abweichung von der Toleranzfeldmitte.}
	 Siehe 7.2.3.png.
	 
	 \subsection{Berechnen Sie den Prozessf\"ahigkeitsindex. Was bedeutet ein Prozessf\"ahigkeitsindex von $c_p$ $\ge$ 1?}
	 Siehe 7.2.4\_prozessf\"ahigkeitsindex.R.
	 Dies bedeutet, dass der Fertigungsprozess einen geringen Ausschu\ss  ($<0,27$ $\%$) aufweist.
	 
	 \subsection{Berechnen Sie den Prozessbrauchbarkeitsindex.}
	 Siehe 7.2.5\_prozessbrauchbarkeitsindex.R.
	 
	 \subsection{Wie hoch ist die Ausschu\ss rate?}
	 Die Ausschu\ss rate betr\"agt $\frac{4}{15} \approx$ 26,7.
	 
	 \section{Aufgabe 3}
	 \subsection{Welcher Genauigkeitsklasse lassen sich die beiden Messger\"ate zuordnen?}
	 
	 \subsection{Welche Varianz erwarten sie f\"ur den pH-Wert?}
	 
	 \subsection{Wie gro\ss ist der maximale relative Fehler des pH-Wertes?}
	
\end{document}