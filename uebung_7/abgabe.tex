\documentclass[12pt, a4paper, titlepage]{article}

\usepackage{ngerman}
\usepackage{lmodern}
\usepackage{amssymb }
\usepackage{subfig}
\usepackage{colortbl}
\usepackage{graphicx}
\usepackage{url} 

\title{Messtechnik und Messdatenverarbeitung \\ \"Ubungszettel 7}
\author{Lasse Knudsen (21157556), Maximilian Scholz (21158423), \\
	Florian Wiesener (21155905)  \\
	Technische Universit\"at Hamburg-Harburg \\
}

\date{\today}
\begin{document}
	\maketitle
	 \section{Aufgabe 1}
	 \label{sec:a1}
	 \subsection{Welchen der behandelten Hypothesentests k\"onnen sie einsetzen, um eine Abl\"osung des Sensors zu
	 	detektieren? Beschreiben Sie kurz wie und unter welchen Voraussetzungen der Test anwendbar ist?}
	
	\section{Aufgabe 2}

	\subsection{Bitte erl\"autern Sie die Bedeutung der Toleranzgrenze.}
	\begin{enumerate}
		\item Da der Mittelwert abh\"angig von zuf\"alligen Stichproben ist, ist er auch eine Zufallsvariable
		\item wahr
		\item wahr
		\item falsch
	\end{enumerate}
	
\end{document}