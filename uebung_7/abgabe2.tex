\documentclass[12pt, a4paper, titlepage]{article}

\usepackage{ngerman}
\usepackage{lmodern}
\usepackage{amssymb }
\usepackage{subfig}
\usepackage{colortbl}
\usepackage{graphicx}
\usepackage{url} 

\title{Messtechnik und Messdatenverarbeitung \\ \"Ubungszettel 7}
\author{Lasse Knudsen (21157556), Maximilian Scholz (21158423), \\
	Florian Wiesener (21155905)  \\
	Technische Universit\"at Hamburg-Harburg \\
}

\date{\today}
\begin{document}
	\maketitle
	 \section{Aufgabe 1}
	 \label{sec:a1}
	 \subsection{Welchen der behandelten Hypothesentests k\"onnen sie einsetzen, um eine Abl\"osung des Sensors zu
	 	detektieren? Beschreiben Sie kurz wie und unter welchen Voraussetzungen der Test anwendbar ist?}
	
	\section{Aufgabe 2}

	\subsection{2.1: Bitte erl\"autern Sie die Bedeutung der Toleranzgrenze.}
	Die Toleranzgrenze gibt an, innerhalb welchen Intervalls um das Sollmaß das Maß eines Werkstücks liegen muss,
	um die gesetzten Qualitätsstandards zu erfüllen.

	\subsection{2.4: Was bedeutet ein Prozessf\"ahigkeitsindex von cp größer / gleich 1?"}
	Dies bedeutet, dass der Fertigungsprozess einen geringen Ausschuß (<0,27 Prozent) aufweist.

	\subsection{2.6: Wie hoch ist die Ausschußrate?"}
	Die Ausschußrate beträgt 4/15 = 26,66.
	
\end{document}