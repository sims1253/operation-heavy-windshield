\documentclass[12pt, a4paper, titlepage]{article}

\usepackage{ngerman}
\usepackage{lmodern}
\usepackage{amssymb }
\usepackage{subfig}
\usepackage{colortbl}
\usepackage{graphicx}
\usepackage{url} 

\title{Messtechnik und Messdatenverarbeitung \\ \"Ubungszettel 7}
\author{Lasse Knudsen (21157556), Maximilian Scholz (21158423), \\
	Florian Wiesener (21155905)  \\
	Technische Universit\"at Hamburg-Harburg \\
}

\date{\today}
\begin{document}
	\maketitle
	 \section{Aufgabe 1}
	 \label{sec:a1}
	 \subsection{Welchen der behandelten Hypothesentests k\"onnen sie einsetzen, um eine Abl\"osung des Sensors zu
	 	detektieren? Beschreiben Sie kurz wie und unter welchen Voraussetzungen der Test anwendbar ist?}
	
	\section{Aufgabe 2}

	\subsection{Bitte erl\"autern Sie die Bedeutung der Toleranzgrenze.}
	Die Toleranzgrenze gibt an, innerhalb welchen Intervalls um das Sollma\ss  das Ma\ss  eines Werkst\"ucks liegen muss,
	um die gesetzten Qualit\"atsstandards zu erf\"ullen.

	\subsection{Berechnen Sie die Standardabweichung und die Abweichung von der Toleranzfeldmitte.}
	Siehe 7.2.2\_abweichung.R.
	
	\subsection{Zeichnen Sie schematisch das Toleranzfeldund die Verteilung deer Bauteill\"angen. Markieren Sie die
		Toleranzfeldmitte und die Abweichung von der Toleranzfeldmitte.}
	Siehe 7.2.3.png.
	
	\subsection{Berechnen Sie den Prozessf\"ahigkeitsindex. Was bedeutet ein Prozessf\"ahigkeitsindex von $c_p$ $\ge$ 1?}
	Siehe 7.2.4\_prozessfaehigkeitsindex.R.
	Dies bedeutet, dass der Fertigungsprozess einen geringen Ausschu\ss  ($<0,27$ $\%$) aufweist.
	
	\subsection{Berechnen Sie den Prozessbrauchbarkeitsindex.}
	Siehe 7.2.5\_prozessbrauchbarkeitsindex.R.

	\subsection{2.6: Wie hoch ist die Ausschu\ss rate?}
	Die Ausschu\ss rate betr\"agt $\frac{4}{15} \approx$ 26,7.
	
\end{document}