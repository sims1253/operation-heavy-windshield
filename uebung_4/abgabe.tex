\documentclass[12pt, a4paper, titlepage]{article}

\usepackage{ngerman}
\usepackage{lmodern}
\usepackage{amssymb }
\usepackage{subfig}
\usepackage{colortbl}
\usepackage{graphicx}
\usepackage{url} 

\title{Messtechnik und Messdatenverarbeitung \\ \"Ubungszettel 4}
\author{Lasse Knudsen (21157556), Maximilian Scholz (21158423), \\
	Florian Wiesener (21155905)  \\
	Technische Universit\"at Hamburg-Harburg \\
}

\date{\today}
\begin{document}
	\maketitle
	 \section{ Optische Transmissionsmessung}
	 \label{sec:opti}
	 \subsection{Zeigen sie, dass es sich bei der gegebenen Verteilung um eine Dichtefunktion handelt.}
	\glqq Die Wahrscheinlichkeitsverteilung (kurz: Verteilung)\grqq \ Puente, Seite 112. Wir gehen von folgender Fragestellung aus:\ \glqq Zeigen sie, dass es sich bei der gegebenen Funktion um eine Dichtefunktion handelt\grqq . 
	\\
	\\
	$ f(x)  =  
	-5(\frac{1}{256} x^4 - \frac{1}{16}) \textmd{ f\"ur } -2 \leq x \leq 2\\
	\textmd{0 sonst}$


	 $$
	 \int_{-2}^2 -5(\frac{1}{256} x^4 - \frac{1}{16} \ dx \ = 
	 -5 \left[\frac{1}{1280}x^5-\frac{x}{16}\right]_{-2}^2 \ = 
	 \ -5\left(\left(\frac{1}{40}-\frac{1}{16}\right)-\left(\frac{-1}{40}+\frac{1}{16}\right)\right) \ = \ 1
	 $$
	 Damit ist gezeigt, dass es sich um eine Dichtefunktion handelt.
	 
	 \subsection{Wie gro\ss \ ist die Wahrscheinlichkeit, dass ein emittiertes Photon detektiert wird?}
	 Aus dem Radius des Detektorelements ergibt sich folgendes Integral:\\
	 $$
	 \int_{-1}^1 -5(\frac{1}{256} x^4 - \frac{1}{16} \ dx \ =
	 -5 \left[\frac{1}{1280}x^5-\frac{x}{16}\right]_{-1}^1 \ = 
	 \ -5\left(\left(\frac{1}{40}-\frac{1}{16}\right)-\left(\frac{-1}{40}+\frac{1}{16}\right)\right)$$ 
	 
	 
	 $$\ = \ 	\frac{79}{128}
	 \ \approx 62 \%
	 $$
	
	\subsection{Angenommen es werden $10^5$ Photonen gez\"ahlt. Wieviele Photonen wurden von der Lichtquelle emittiert?}
	\begin{eqnarray}
	\frac{79}{128} \ &=& \ 10^5 \\
	\frac{1}{128} \ &=& \ \frac{10^5}{79}\\
	1 \  \ & \approx & 162025
	\end{eqnarray}

	\subsection{Bitte geben sie die gemeinsame Wahrscheinlichkeitsdichtefunktion an.}
	Da $f_x (x)$ und $f_y (y)$ statistisch unabh\"angig sind ergibt sich die gemeinsame Dichtefunktion durch das Multiplizieren der einzelnen Funktionen:
	$$f_{x,y}(x,y)= -5(\frac{1}{256} x^4 - \frac{1}{16}) \cdot \frac{1}{\mu}\cdot e^{\frac{-y}{\mu}}$$
	W\"aren $f_x (x)$ und $f_y (y)$ nicht mehr statistisch unabh\"angig, könnte die gemeinsame Wahrscheinlichkeitsdichtefunktion nicht mehr durch einfaches Multiplizieren berechnet werden.
	
	\section{Stichprobe}
	Siehe 3\_gesamt.R
	\subsection{Ergebnis}
	\begin{enumerate}
		\item Da der Mittelwert abh\"angig von zuf\"alligen Stichproben ist, ist er auch eine Zufallsvariable
		\item wahr
		\item wahr
		\item falsch
	\end{enumerate}
	
\end{document}